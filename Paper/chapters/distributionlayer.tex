\documentclass[12pt,a4paper]{article}
\usepackage[utf8]{inputenc}
\usepackage[ngerman]{babel}
\usepackage[T1]{fontenc}
\usepackage{hyperref}
\usepackage{todonotes}

\newcommand{\code}[1]{\texttt{#1}}
\definecolor{codegray}{gray}{0.9}
\newcommand{\graycode}[1]{\colorbox{codegray}{\texttt{#1}}}

\newcommand{\myparagraph}[1]{\paragraph{#1}\mbox{}\\}
\newcommand{\mysubparagraph}[1]{\subparagraph{#1}\mbox{}\\}


\begin{document}

To ensure simple and efficient data lookups CockroachDB provides a monolithic sorted map of key-value pairs. Those key-value
pairs include two fundamental elements. One stores the meta range and the other the table data.

\todo{image which shows example simplified key values}

\paragraph{Meta ranges}
describe the location of data including all of its replicas in the cluster.

\paragraph{Table data}
describes the rows of a table included in this particular range.

\end{document}

